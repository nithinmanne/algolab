\documentclass[a4paper,11pt]{article}
\usepackage[utf8]{inputenc}
\usepackage{times}

\title{Report for assignment 4}
\author{Manne Naga Nithin (14EE10026)}

\begin{document}

\maketitle

\paragraph{Finding the Last Person Standing}
\begin{enumerate}
 \item \textbf{First step}\newline
The input is given as the number of people present.

 \item \textbf{Second step}\newline
 We pass the number of elements remaining, the first element that was not yet written off, and the difference between two consecutive element which doubles every time we pass the function recursively. We go in a loop printing every alternate element till we reach the end. Each value is equal to the previous value plus the difference. At the end of the loop, if the last value was skipped, we also print the first element and update it. We then pass the updated first value, double the difference and the number of values remaining.

 \item \textbf{Analytical Solution}\newline
 We know by observation that\newline
 if n is even, $V(n) = 2V(n/2)-1$\newline
 if n was odd, $V(n) = 2V(n/2)+1$\newline
 The given result is :\newline
 $V(2^m+l)=2l+1$\newline
 We can prove this using Mathematical Induction. The base case $n=1$ is true.We then consider separately for even and odd cases.\newline
 If n is even,\newline
 we choose $l_1$ and $m_1$ such that $n/2 = 2^{m_1}+l_1$ and $0\leq l_1 < 2^{m_1}$.\newline
 Note that $l_1 = l/2$.\newline
 We have $f(n) = 2f(n/2)-1=2((2l_1)+1)-1 = 2l+1$, where the second equality follows from induction.\newline
 If n was odd,\newline
 we choose $l_1$ and $m_1$ such that $(n-1)/2 = 2^{m_1}+l_1$ and $0\leq l_1 < 2^{m_1}$.\newline
 Note that $l_1 = (l-1)/2$.\newline
 We have $f(n) = 2f((n-1)/2)+1=2((2l_1)+1)+1 = 2l+1$, where the second equality follows from induction.\newline
 Hence proved for both odd and even using Mathematical Induction .\newline
 If we solve for n, $f(n) = 2(n-2^{\lfloor \log_2(n) \rfloor})+1$
\end{enumerate}
\end{document}
