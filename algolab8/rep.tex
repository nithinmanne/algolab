\documentclass[a4paper,11pt]{article}
\usepackage[utf8]{inputenc}
\usepackage{times}

\title{Report for assignment 8}
\author{Manne Naga Nithin (14EE10026)}

\begin{document}

\maketitle

\paragraph{}
\begin{enumerate}
 \item \textbf{Pseudo Code}\newline
 $temp=heap.hp[1]$\newline
 If time in temp is higher than given time, break\newline
 If invalid, goto beginning\newline
 $heap=deletemin(heap)$\newline
 Loop for updating positions\newline
 $fsave(STATE,temp,prep)$ Saving to File\newline
 $vcol(STATE,temp.a,temp.b)$ Collision Velocity Calculation if a and b are colliding\newline
 Using $heap=insert(heap,col)$ for adding new collisions into heap\newline
 goto beginning\newline
 
 \item \textbf{Key Operations and the heap structure}\newline
 The key operations are :\newline
 $heap=insert(heap,col)$\newline
 $heap=deletemin(heap)$\newline
 $insert$ inserts an element into the heap\newline
 $deletemin$ deletes the minimum element in the heap\newline
 The data structure for the variable $heap$ contains :\newline
 $heapnode\ *hp$\newline
 $int\ n$\newline
 $n$ corresponds to the size of the heap\newline
 $*hp$ corresponds to an array which is dynamically allocated later\newline
 The heap is essentially an array\newline
 The data structure for the variable $hp$ contains :\newline
 Details of 2 colliding particles\newline
 Time of collision\newline
 
 \item \textbf{Time Complexities of Heap Operations}\newline
 $insert$ is done by simply placing the element at end and percolating upwards\newline
 Time Complexity = $O(log(n))$\newline
 $deletemin$ removes the top node,replaces it with last node and percolates down\newline
 Time Complexity = $O(log(n))$\newline
 
  \item \textbf{Time Complexity of Complete Simulation Run}\newline 
  $deletemin$ happens only once\newline
  $insert$ happens at most 8 times for each of the colliding balls with remaining three balls and with a wall\newline
  Time Complexity = $O(log(n))$\newline
 

\end{enumerate}


\end{document}
