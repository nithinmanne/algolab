\documentclass[a4paper,11pt]{article}
\usepackage[utf8]{inputenc}
\usepackage{times}

\title{Report for assignment 1}
\author{Manne Naga Nithin (14EE10026)}

\begin{document}

\maketitle

\paragraph{Multiplying two very large numbers}
\begin{enumerate}
 \item \textbf{First step}

We have to make the size a power of 2, since fft is only applicable for powers of 2.

This can be obtained by running a loop till the number given is less than a power of 2.

 \item \textbf{Second step}

An inline formula: $\mathbf{F(C)} = \mathbf{F(A)} \mathbf{F(B)}$.

We can calculate the fft of A and B using the recursive fft function, by dividing the array into two parts even and odd, sinse their size is a power of 2, this will not be a problem.

 \item \textbf{Third step}
 
 We multiply all the elements in both the arrays to get the third array.
 We can calculate the inverse Fourier of the resultant array to get C.

\end{enumerate}


\paragraph{Recurrence to characterise the running time}

The time required is in the form of
 $nlog(n)$.

\end{document}
