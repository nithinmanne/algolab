\documentclass[a4paper,11pt]{article}
\usepackage[utf8]{inputenc}
\usepackage{times}

\title{Report for assignment 10}
\author{Manne Naga Nithin (14EE10026)}

\begin{document}

\maketitle
\paragraph{Currency Exchange}
\begin{enumerate}

 \item \textbf{Solution}\newline
 We can solve this problem by finding a negative cycle in the currency conversion table. 
 This means that there is a path which leads back to the same currency but we end up with more than what we started. 
 This maps to the Bellman-Ford Algorithm. Here, we need to multiply instead of add or subtract, so we can take the negative logarithm of the conversion table, and implement this algorithm directly to find a negative cycle if one exists.\newline
 \item \textbf{Bellman-Ford Algorithm}\newline
 We can represent this in form of a graph with the currencies as the vertexes and the exchange rates as the lines between the respective currencies.
 We take two arrays, one containing exchange rates and the other the distance from the first currency. This is initialized with infinity for all but the first which is 0.
 Now  relax edges repeatedly for all currencies times in a loop for each currency set (u,v) with rate w in edges by updating the minimum distance from the first one to the current one. 
 Now check for negative cycles by checking the condition distance $ex.rate\ to\ u\ +\ w\ <\ ex.rate\ to\ v$ for all currencies.\newline
 \item \textbf{Time Complexity}\newline
 The time complexity for Bellman-Ford Algorithm is in the order of vertexes multiplied by the edges. 
 Here the number of edges is the number of currencies squared.\newline
 Hence, the total running time is $O(n^3)$.
 
 
 


\end{enumerate}


\end{document}
