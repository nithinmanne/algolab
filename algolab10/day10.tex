\documentclass[a4paper,11pt]{article}
\usepackage[utf8]{inputenc}
\usepackage{times}

\title{Report for assignment 10}
\author{Mekala RajasekharReddy (14EE10027)}

\begin{document}

\maketitle

\paragraph{Currency Exchanges:}
\begin{enumerate}
 \item \textbf{procedure:}
If we consider this problem by starting conversion from say A and return back to A via X lastly,then this problem is equivalent  to finding if the conversion from A to X is same in any possible way from A to X  with direct conversion.\\
 Thus if W1,W2,W3...Wn are the weights of conversion edges,then we want to find a loop such that w1.W2.W3..Wn>1.\\
 =>  log(W1)+log(W2)+.....log(Wn)>0\\
 =>   log(1/W1)+log(1/W2)+.......log(1/Wn)<0.
Now if we convert the initial money via a logarithmic scale then, multiplication along two currencies becomes addition.\\ 
This algorithm turns down to finding a negative loop in a graph and Bellman-Ford Algorithm provides the way for This.
 \item \textbf{Algorithm :}
This implementation takes in a graph, represented as lists of vertices and edges, and fills two arrays  i.e,distance with shortest-path.
 At the beginning , all vertices have a weight of infinity(10,000 here)except for the Source, where the Weight is zero.\\
 Now  relax edges repeatedly for vertices-1 times in a loop for each edge (u,v) with weight w in edges by updating the minimum distance from A to X .\\
 Now check for negative-weight cycles by checking the condition distance u + w $<$ distance v for all vertices.
Thus the running time is  O($n^3$).\\
 \item \textbf{Running Time :}
In this function,once we  take the table the program has to be run for all edges for V-1 times and we have to check the condition for negative loop.\\
Hence the running time of the function is V *E where V is the number of vertices and  E is the Number of Edges,In this case the edges are of $\ominus$($n^2$). \\

\end{enumerate}
\end{document}